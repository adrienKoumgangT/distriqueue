\chapter{Introduction}\label{ch:introduction}


TaskForce is a distributed job scheduling system.
The system demonstrates the practical application of fundamental distributed systems concepts including consensus algorithms,
message-oriented middleware, fault tolerance, and distributed coordination.

The project implements a complete job scheduling pipeline with:
\begin{itemize}
    \item \textbf{Erlang/OTP orchestrator}\cite{erlang} using Raft consensus for leader election
    \item \textbf{RabbitMQ cluster}\cite{rabbitmq} with priority-based queuing and dead letter handling
    \item \textbf{Java Spring Boot API gateway}\cite{springboot} with REST endpoints and real-time SSE updates
    \item \textbf{Polyglot worker farm} supporting Python and Java job execution
    \item \textbf{Comprehensive monitoring} with Prometheus and Grafana
    \item \textbf{Multi-node deployment} on Docker containers and Ubuntu VMs
\end{itemize}

The system successfully handles concurrent jb submissions, automatically recovers from node failures,
maintains exactly-once execution semantics, and scales horizontally by adding worker nodes.
All course requirements regarding synchronization/coordination, communication, Erlang implementation,
and multi-node deployment have been fully satisfied.


\section{Background}\label{sec:background}

Distributed systems have become the backbone of modern computing infrastructure.
From cloud computing platforms to microservices architectures,
the ability to coordinate work across multiple nodes while maintaining consistency and availability is fundamental.
Job scheduling - the allocation of computational tasks to worker nodes - represents a classic distributed systems problem
that encompasses many core challenges: coordination, communication, fault tolerance, and load balancing.

\section{Problem Statement}\label{sec:problem-statement}

Modern distributed applications require reliable job processing systems that can:
\begin{itemize}
    \item Handle thousands of concurrent task submissions
    \item Guarantee exactly-once or at-least-once execution semantics
    \item Tolerate node failures without data loss
    \item Scale horizontally as workload increases
    \item Support heterogeneous execution environments
    \item Provide real-time visibility into system state
\end{itemize}

Existing solutions like Apache Airflow, Celery, and AWS Batch are powerful but complex.
This project aims to build a simplified yet complete distributed job scheduler from first principles,
demonstrating how fundamental distributed systems algorithms and middleware technologies address these challenges.

\section{Project Scope}\label{sec:project-scope}

The scope of this project includes:
\begin{itemize}
    \item Design and implementation of a distributed job scheduler
    \item Implementation of Raft consensus algorithm for leader election
    \item Integration with RabbitMQ for reliable message queuing
    \item Development of a polyglot worker execution framework
    \item REST API for job submission and management
    \item Real-time status streaming via Server-Sent Events
    \item Multi-node deployment on both containers and VMs
    \item Comprehensive monitoring and observability
\end{itemize}



